\documentclass[reprint, amsmath, amssymb, aps]{revtex4-2}

\usepackage{graphicx}
\graphicspath{{./assets/figures/}}
\usepackage{dcolumn}
\usepackage{bm}
\usepackage{diagbox}
\usepackage[table]{xcolor}
\usepackage{hyperref}
\usepackage{comment}
\newcolumntype{L}{>{$}l<{$}} % math-mode version of "l" column type
\usepackage{float} % To fix location of tables with H
\usepackage[ruled,vlined]{algorithm2e}
\usepackage{braket}
\usepackage{amsthm}
\usepackage{cleveref}
\usepackage{mathtools}
\usepackage{listings} % to add python code
\usepackage{qcircuit}

\newtheorem{definition}{Definition}
\newtheorem{prop}{Proposition}
\DeclareMathOperator{\tr}{Tr}

\begin{document}

\preprint{}

\title{Long-range intermolecular interactions on a photonic quantum computer}
%\thanks{}

\author{Matthieu Sarkis}
\email{matthieu.sarkis@uni.lu}
\affiliation{
Department of Physics and Materials Science\\ University of Luxembourg, L-1511, Luxembourg City, Luxembourg.
}

\author{Alessio Fallani}
\email{alessio.fallani@uni.lu}
\affiliation{
Department of Physics and Materials Science\\ University of Luxembourg, L-1511, Luxembourg City, Luxembourg.
}

\author{Alexandre Tkatchenko}
\email{alexandre.tkatchenko@uni.lu}
\affiliation{
Department of Physics and Materials Science\\ University of Luxembourg, L-1511, Luxembourg City, Luxembourg.
}

%\collaboration{}%\noaffiliation

\date{\today}

\begin{abstract}
%\begin{description}
%\item[Usage]
%Secondary publications and information retrieval purposes.
%\item[Structure]
%You may use the \texttt{description} environment to structure your abstract;
%use the optional argument of the \verb+\item+ command to give the category of each item.
%\end{description}
\end{abstract}

%\keywords{Suggested keywords}%Use showkeys class option if keyword
                              %display desired
\maketitle

%\tableofcontents

\section{Introduction}

    Dispersion forces, also known as van der Waals (vdW) forces originate from the electromagnetic interaction between electrically neutral atoms or molecules which do not have permanent electric moments \cite{margenau2013theory,kaplan2006intermolecular,stone2013theory,hirschfelder2009intermolecular}. They are ever-present long-range forces between atoms or molecules arising from the zero-point fluctuations of the quantum electromagnetic field \cite{casimir1948influence,buhmann2013dispersion,buhmann2007dispersion,compagno1995atom,passante2018dispersion}. Their importance can be appreciated on both macro- and micro-scales; for instance, the first macroscopic signature of dispersion forces is the well-known correction to the equation of state of an ideal gas that led to the van der Waals equation \cite{milonni2013quantum}. Moreover, dispersion forces also influence the structure of liquids and solids such as the anomalies of water \cite{schmid2001recent} as well as the macroscopic properties of macromolecules such as their structure \cite{hoja2019reliable}, stability \cite{hoja2018first,mortazavi2018structure}, dynamics \cite{stohr2019quantum,reilly2014role,galante2021anisotropic}, and electric \cite{kleshchonok2018tailoring} and optical \cite{ambrosetti2022optical} responses. The most natural framework for the investigation of the response of matter subjected to such forces is quantum electrodynamics \cite{cohen1997photons,cohen1998atom,bookpreparata,salam2009molecular,craig1998t}. However, as widely shown in the literature, the inclusion of vdW dispersion interactions can be done by means of many-body methods \cite{richardson1975dispersion,mahanty1973dispersion,woods2016materials,tkatchenko2015current,ren2012random,harl2009accurate,dobson2012calculation,parsegian2005van}. Dispersion vdW interactions are often represented within the Lennard-Jones approach, namely, through a pairwise two-body interatomic potential \cite{becke2006simple,becke2006exchange,grimme2010consistent,grimme2006semiempirical,tkatchenko2012accurate,massa2021many} of the form $C_{6}/R^{6}$ (where $R$ is the interatomic distance and $C_6$ a system-dependent constant). Among all the existing models in the literature, the many-body dispersion (MBD) framework has been undoubtedly proved to be an accurate approach \cite{tkatchenko2012accurate,ambrosetti2014long}. In the MBD framework, the drudonic response of valence drudons in atoms and molecules is supposed to be linear and this can be formally done through the introduction of the quantum Drude oscillator (QDO). A single QDO is coarse-grained quantum-mechanical model in which the properties of an atom are encompassed in a small number of parameters. The model consists in assimilating the atom to a point particle of mass $m$ and electric charge $-q$ attached to a fixed (infinite mass) center of charge $+q$ by a harmonic spring characterized by a frequency $\omega$. Molecules are then defined as a collection of QDOs in dipole-dipole interaction.
    For some specific choices of the matter system geometry, the quantum Hamiltonian can be exactly diagonalized. For instance, in a closed linear chain of molecules, one can analytically solve for the spectrum of the Hamiltonion \cite{doi:10.1063/1.1743992}. In the case of a general geometry, instead, one can solve for the London-van der Waals interaction energy through a perturbative approach \cite{doi:10.1063/1.1743991}.
    This simple model has been extensively used in various contexts, for instance in order to tackle the drudonic structure problem for isolated molecules, in particular long range interactions, as well as to study the impact of an ambient bath or of an external electric field on molecular properties \cite{Karimpour_2022, karimpour2021comprehensive}.
    Though simple, through a numerical treatment this system was shown to capture long-range phenomena in large biomolecular systems \cite{https://doi.org/10.48550/arxiv.2205.11549}. By construction, the MBD framework relies on the dipole approximation of the drudon–drudon Coulomb interaction leaving aside any contribution coming from high-order terms. In the literature, dispersion forces have been addressed mostly for atomic dimers and small systems, via multipolar generalizations of the pairwise second-order perturbative approaches \cite{massa2021beyond,massa2021many,becke2006simple,becke2006exchange}.
    In this Letter ...
    \textcolor{purple}{They did full coulomb FCI in \cite{sadhukhan2016quantum}}

    \textcolor{purple}{
        \begin{itemize}
            \item Reduce the part about MDE/QDOs
            \item Write the part about quantum computing/photonics/VQE
        \end{itemize}
    }

    \begin{figure}
    \label{fig:qdos}
        %\hspace{1.25cm}
        %\begin{subfigure}
        \includegraphics[scale=0.4]{figures/qdos.png}
        %\end{subfigure}
        \caption{Illustration of a system composed of a pair of QDOs. The nuclei are considered to be infinitely massive. The drudons interact with their nucleus through a harmonic potential, but with the other QDO through Coulomb interaction. $\bm x_i$ denote the relative position of the drudon with respect to its nucleus in QDO $i$. $\bm r$ denotes the position of $\text{QDO}_1$ with respect to $\text{QDO}_2$.}
    \end{figure}


\section{Definition of the model}

    \subsection{Three-dimensional model}

        The Hamiltonian describing a system of $N$ Quantum Drude Oscillators in three dimensions is given by:
        \begin{equation}
        \label{eq:full_QDO_Hamiltonian}
            H=\sum_{i=1}^N\left[\frac{\bm{p} _i^2}{2m_i} + \frac{1}{2}m_i\omega_i^2\bm{x} _i^2\right] +\sum_{i<j}V_\text{Coul}\left(\bm{x} _i, \bm{x} _j\right)\,,
        \end{equation}
        with the Coulomb interaction receiving contributions from interacting drudon-drudon and drudon-nucleus pairs:
        \begin{equation}
        \label{eq:full_coulomb_potential}
        \begin{split}
            \frac{V_\text{Coul}\left(\bm{x} _i, \bm{x} _j\right)}{q_iq_j}=&\ \frac{1}{r_{ij}} - \frac{1}{|\bm{r}_{ij}
            + \bm{x} _i|} - \frac{1}{|\bm{r}_{ij}  - \bm{x} _j|} \\
            & + \frac{1}{|\bm{r}_{ij} - \bm{x} _j + \bm{x} _i|}\,.
        \end{split}
        \end{equation}
        The usual approach consists in solving the theory in the multipolar expansion framework, in which the potential can be expressed as a power series in the inverse distance separating the two centers:
        \begin{equation}
            V_\text{Coul}\left(\bm{x} _i, \bm{x} _j\right)= \sum_{n\geq 0} V_n\left(\bm{x} _i, \bm{x} _j\right)\,,
        \end{equation}
        with the following scaling behavior in terms of the distance between the centers:
        \begin{equation}
            V_n\left(\bm{x} _i, \bm{x} _j\right)\propto r_{ij}^{-n-3}\,.
        \end{equation}
        The potential $V_0$ corresponds then to the dipole-dipole interaction, and is at the core of the Many Body Dispersion (MBD) model. $V_1$ corresponds to the dipole-quadrupole interaction, and $V_2$ to the quadrupole-quadrupole and dipole-octupole interaction.

        One obvious limitation of the multipolar expansion is the lower bound it imposes on the interatomic distance. One can easily see that within the MBD (dipole-dipole) model by direct diagonalization of the quadratic Hamiltonian in terms of normal modes. In that case, one of the normal modes (the center of mass mode) develops a purely imaginary frequency at short range. Higher order physical effects are also neglected in the MBD model, motivating the study of the QDO model with full Coulomb interaction potential between its constituents.

        Let us define the following dimensionless position and momententum operators associated to QDO $i$:
        \begin{equation}
            \bm{X}_i := \sqrt{\frac{m_i\omega_i}{\hbar}}\,\bm{x}_i\,,\ \ \ \ \ \bm{P}_i := \frac{\bm{p}_i}{\sqrt{\hbar m_i\omega_i}}\,,
        \end{equation}
        in terms of which the Hamiltonian reads
        \begin{equation}
        \begin{split}
            H =&\ \sum_{i=1}^N\frac{\hbar\omega_i}{2}\left(\bm X_{i}^2 + \bm P_{i}^2\right) \\
            & + \sum_{i<j}V_\text{Coul}\left(\sqrt{\frac{\hbar}{m_i\omega_i}}\bm{X} _i, \sqrt{\frac{\hbar}{m_j\omega_j}}\bm{X} _j\right)\,.
        \end{split}
        \end{equation}
        One can define the $3N$ creation and annihilation operators
        \begin{equation}
            \bm a_{i} = \frac{\bm X_{i} + i\bm P_{i}}{\sqrt 2}\,,\ \ \ \ \ \bm a^\dagger_{i} = \frac{\bm X_{i} - i\bm P_{i}}{\sqrt 2}\,,
        \end{equation}
        in terms of which the Hamiltonian reads
        \begin{equation}
        \begin{split}
            &H = \sum_{i=1}^N\hbar\omega_i\left(\bm a_{i}^\dagger\cdot\bm a_{i} +\frac{3}{2}\right) \\
            & + \sum_{i<j}V_\text{Coul}\left(\sqrt{\frac{\hbar}{m_i\omega_i}}\frac{\bm a_i + \bm a_i^\dagger}{\sqrt 2}, \sqrt{\frac{\hbar}{m_j\omega_j}}\frac{\bm a_j + \bm a_j^\dagger}{\sqrt 2}\right)\,.
        \end{split}
        \end{equation}
        Let us from now on restrict the problem to a pair of QDOs ($N=2$) for concreteness, though the following developements carry to bigger systems. We denote by $d$ the interatomic distance.

    \subsection{One-dimensional case}

        In order to reduce the complexity of the problem, let us define one-dimensional instances of the QDO model as follows: we restrict the movement of the two drudons to be along a common axis (directed by a unit vector $\hat{\bm e}_\theta$) which form an angle $\theta\in[0,\pi/2]$ with respect to the vector $\bm r_{12}$ connecting the two nuclei. We therefore have a family of one-dimensional models which can be obtained from the full-fledged 3d model simply by setting to zero the contribution from the oscillator modes belonging to the plane perpendicular to $\hat{\bm e}_\theta$. Let us denote by $(X, P)$ the remaining position and momentum degree of freedoms. As limiting cases, we obtain models in which the drudons are constrained to move either in the direction parallel to the axis separating the two nuclei ($\theta=0$), or perpendicular to the latter ($\theta=\pi/2$). Those two models were studied in \cite{anderson2022coarse} in the dipole-dipole approximation. In that paper, the authors encode the states in the truncated Fock space of the system into the state of a set of qubits, and run VQE-type algorithms on IBQ quantum processors. However as we will see, the angle $\theta$ captures the competition between \textit{existence of binding} (for small $\theta$) and \textit{smoothness} (for large $\theta$), and interesting one-dimensional model actually sit at values of $\theta$ in the open segment $(0,\pi/2)$.

        In the case of a generic angle $\theta$ and interatomic distance, the one-dimensional Coulomb potential reads:
        \begin{equation}
        \begin{split}
            &\frac{V_\text{Coul}^{\theta, d}(x_1, x_2)}{q_1q_2} = \frac{1}{d} - \frac{1}{\sqrt{d^2+2d(\cos\theta) x_1+x_1^2}} \\
            &- \frac{1}{\sqrt{d^2-2d(\cos\theta) x_2+x_2^2}} \\
            &+\frac{1}{\sqrt{d^2 - 2d(\cos\theta) (x_2-x_1) + (x_2-x_1)^2}}\,.
        \end{split}
        \end{equation}
        We therefore have a family of Hamiltonians parameterized by $(\theta, d)\in[0, \pi/2]\times\mathbb R_{>0}$:
        \begin{equation}
        \label{eq:finalHamiltonian}
        \begin{split}
            H_{\theta, d}=&\ \frac{\hbar\omega_1}{2}\left(P_1^2 + X_1^2\right) + \frac{\hbar\omega_2}{2}\left(P_2^2 + X_2^2\right)\\
            &+V_\text{Coul}^{\theta, d}\left(\sqrt{\frac{\hbar}{m_1\omega_1}}X_1, \sqrt{\frac{\hbar}{m_2\omega_2}}X_2\right)\,.
        \end{split}
        \end{equation}
        Alternatively, in terms of creation and annihilation operators, one has
        \begin{equation}
        \label{eq:Hamiltonian_heterodyne}
        \begin{split}
            H_{\theta, d} &= \hbar\omega_1\left( a_{1}^\dagger a_{1} +\frac{1}{2}\right) + \hbar\omega_2\left(a_{2}^\dagger a_{2} +\frac{1}{2}\right) \\
            & + V_\text{Coul}^{\theta, d}\left(\sqrt{\frac{\hbar}{m_1\omega_1}}\frac{a_1 + a_1^\dagger}{\sqrt 2}, \sqrt{\frac{\hbar}{m_2\omega_2}}\frac{a_2 + a_2^\dagger}{\sqrt 2}\right)\,.
        \end{split}
        \end{equation}
        For the numerical simulations, we set $\hbar=4\pi\epsilon_0=1$ as well as $m_i=q_i=\omega_i=1$ for both QDOs.

\section{Photonic circuit and Variational Algorithm}

    The main point of this letter is to probe the very natural idea that photonic-based continuous variable quantum hardware should be perfectly suited to simulate bosonic degrees of freedom. The simplest example one can come up with is a finite collection of interacting bosonic harmonic oscillators, which is precisely the case for the Quantum Drude Oscillator model. Another application is the computation of correlation functions of bosonic quantum fields on a lattice, for which an exponential speedup can be achieved with respect to standard methods \cite{marshall2015quantum}.

    From a molecular physics and long-range intermolecular perspective, we are mainly interested in knowing the ground state of the system (\ref{eq:finalHamiltonian}). Among the various approaches, the mixed quantum-classical variational algorithms were shown to be particularly efficient at capturing the properties of potentially complicated quantum states. In the spirit of \cite{killoran2019continuous, arrazola2019machine}, we apply a continuous-variable version of the variational quantum eigensolver algorithm to extract the ground state of the QDO system. The ground state is obtained by optimizing the parameters of a parameterized optical circuit composed of linear multi-modes gates (two-mode beamsplitters and rotation gates), Gaussian gates (single-mode squeezing and displacement gates), as well as non-Gaussian gates (Kerr gates) implementing non-linearity. Multiple layers constructed out of these gates are stacked to increase the complexity of the ansatz space, cf. fig. (\ref{fig:quantum_circuit}).

    \begin{figure}
    \label{fig:quantum_circuit}
        \begin{equation*}
            \Qcircuit  @C=1em @R=1em {
            \lstick{\ket{0}}& \multigate{1}{BS}  & \gate{R}  & \gate{S}  & \multigate{1}{BS}  & \gate{R}  & \gate{D}  & \gate{K}  & \qw \gategroup{1}{5}{2}{6}{.7em}{--} \\
            \lstick{\ket{0}}& \ghost{BS}  & \qw  & \gate{S}  & \ghost{BS}  & \qw  & \gate{D}  & \gate{K} &
         \qw \gategroup{1}{2}{2}{3}{.7em}{--}
         \\
        }
        \end{equation*}
        \caption{One layer in the optical quantum circuit. The various gates, namely the beamsplitters, rotation, squeezing, displacement and Kerr gates are collectively parameterized by $\omega$.}
    \end{figure}

    In the procedure, the relative coordinate of the drudons with respect to their nucleus is directly identified with the position quadrature of the quantized electromagnetic field along the two channels of the optical quantum circuit.

    Overall, denoting by $\omega$ the set of all parameters, the circuit implements a unitary $U(\omega)$ acting on an input reference state that we simply take to be the Fock vacuum state $|0\rangle$. The state prepared by the circuit is therefore given by
    \begin{equation}
        |\psi(\omega)\rangle = U(\omega)|0\rangle\,.
    \end{equation}

    Once the ansatz state $|\psi(\omega)\rangle$ has been produced, one extracts the value of the energy in that state, given by
    \begin{equation}
        \langle\psi(\omega)|H|\psi(\omega)\rangle\,.
    \end{equation}
    Let us denote expectations in the state $|\psi(\omega)\rangle$ by angular brackets $\langle\cdot\rangle$.
    To be specific, let us take the model $H$, and let us denote by angular brackets the expectation in state $|\psi(\omega)\rangle$. One has
    \begin{equation}
    \label{eq:loss}
    \begin{split}
        \langle H\rangle =&\ \hbar\omega_1\left(\langle a_{1}^\dagger a_{1}\rangle+\frac{1}{2}\right)+\hbar\omega_2\left(\langle a_{2}^\dagger a_{2}\rangle+\frac{1}{2}\right)\\
        &+ \left\langle V_\text{Coul}^{\theta, d}\left(\sqrt{\frac{\hbar}{m_1\omega_1}} X_1, \sqrt{\frac{\hbar}{m_2\omega_2}} X_2\right)\right\rangle
    \end{split}
    \end{equation}
    by linearity of the expectation. On the second line one has to compute an expression of the form $\langle f(X_1, X_2)\rangle$. One therefore needs to extract the joint statistics of the position quadratures by preparation and measurements of the state $|\psi(\omega)\rangle$ in the quadrature basis, as summarized in alg. (\ref{alg:statistics_computation}). Once the joint probability density $\rho$ of $(X_1, X_2)$ in the state $|\psi(\omega)\rangle$ is known, one can compute
    \begin{equation}
        \langle f(X_1, X_2)\rangle = \int_{\mathbb R^{6}}f(x_1, x_2)\rho(x_1, x_2)\,\text{d}x_1\text{d}x_2\,,
    \end{equation}
    where the integral above should be understood a finite sum over a sufficiently refined grid $G_X\times G_X$ in the position quadratures plane. For the numerical simulation we take a linear grid $G_X$ composed of 500 points in the interval $[-6,6]$.

    Let us note that alternatively, and following \cite{arrazola2019machine}, the simulator of Strawberry Fields actually provides direct access to the output statevector expressed in the Fock basis:
    \begin{equation}
        |\psi(\omega)\rangle = \sum_{n_1,n_2=0}^\infty \alpha_{n_1n_2}(\omega)|n_1\rangle\otimes|n_2\rangle\,.
    \end{equation}
    The Fock space is actually truncated at a some fixed energy level, we choose that cutoff level to be 5.
    The amplitude of a specific pair of the quadratures $(X_1, X_2)$ is then given by:
    \begin{equation}
        \langle X_1,X_2|\psi(\omega)\rangle = \sum_{n_1,n_2=0}^\infty \alpha_{n_1n_2}(\omega)\prod_{i=1}^{2}\frac{e^{-\frac{X_i^2}{2}}H_{n_i}(X_i)}{\sqrt{\pi^{1/2}2^{n_i}n_i!}}\,,
    \end{equation}
    in terms of the Hermite polynomials. The joint law of the quadratures in the state $|\psi(\omega)\rangle$ is therefore given by
    \begin{equation}
        \rho(X_1,X_{2}) = \left|\langle X_1,X_2|\psi(\omega)\rangle\right|^2\,.
    \end{equation}
    After extracting as well the mean photon numbers $\langle n_1\rangle$ and $\langle n_2\rangle$, one obtains $\langle H\rangle$.

    We then define the cost function
    \begin{equation}
    \label{eq:definition_cost}
        \mathcal C(\omega) := \langle\psi(\omega)|H|\psi(\omega)\rangle\,,
    \end{equation}
    and update the parameters $\omega$ of the optical circuit in order to minimize that cost. This is summarized in alg. (\ref{alg:training})

    \begin{algorithm}
        \caption{Extract distribution of position quadratures}\label{alg:statistics_computation}
            \textbf{Parameters:} statevector $|\psi\rangle$, finite subset $\mathcal {G}\in\mathbb R$, shots $M\in\mathbb N$

            \KwResult{Probability distribution of $(X_1, X_2)$ in state $|\psi\rangle$ discretized over the grid $G_X\times G_X$}\

            \For{$m=1$ to $|\mathcal G|$}{
                Initialize $N_m \gets 0$;
            }
            \For{$j=1$ to $M$}{
            Measure the state $|\psi\rangle$ in the discretized position quadratures basis, obtain the eigenvalue $X_m$, set $N_m\gets N_m+1$\;
            }
            \For{$m=1$ to $|\mathcal G|$}{
                Normalize $N_m \gets N_m / M$;
            }
            \textbf{return} $\{N_m\}_{m=1}^M$.
    \end{algorithm}



    \begin{algorithm}
        \caption{Training of the parameterized photonic circuit}\label{alg:training}
        \textbf{Parameters:} Model $(\theta, d)$, $N_\text{steps}\in\mathbb N$, initial circuit parameters $\omega_0\in\mathbb R^K$, learning rate $\eta\in\mathbb R_+$

        \KwResult{Optimized hyperparameters $\omega\in\mathbb R^K$}\

        Initialize hyperparameters $\omega \gets \omega_0$\;
        \For{$i=1$ to $N_\text{steps}$}{
        Compute the loss $\mathcal C$ according to eq. (\ref{eq:definition_cost})\;
        Compute the gradient $\nabla_\omega\mathcal C$ with the shift rule\;
            Update the parameters $\omega \gets \omega - \eta\nabla_\omega\mathcal C$\;
        \textbf{end for}
        }
        \textbf{return} $\omega$.
    \end{algorithm}

    As a side remark, let us note that given the form of the Hamiltonian in terms of creation and annihilation operators (\ref{eq:Hamiltonian_heterodyne}), an alternative to the measurement of the position quadratures and photon number operators could be to measure to perform a measurement in the coherent basis through heterodyne measurements.

\section{Results and interpretation}

    We gather here the results of the simulations. We focus on the case of 2 QDOs. In particular we study the profile of the binding energy as a function of the distance between the two nuclei, and make a few observation about the behavior of the entanglement entropy of the system.

\subsection{Binding energy curve}

    We fix a grid $G_\theta\times G_d\subset [0, \pi/2]\times (0, 3.5]$, with $\text{card}(G_\theta)=20$ and $\text{card}(G_d)=200$. For each pair $(\theta, d)$ in the grid we perform the continuous-variables VQE algorithm to extract properties of the ground state of the Hamiltonian (\ref{eq:finalHamiltonian}). Let us denote by $|\psi_{\theta, d}\rangle$ the corresponding ground state.
    The binding energy is defined as the difference between the ground state energy of the system interacting QDOs and the ground state energy of a system of uninteracting QDOs $H_0$, i.e. with electric charge turned off:
    \begin{equation}
        E^\theta_b(d) = \langle\psi_{\theta, d}|H_{\theta, d}|\psi_{\theta, d}\rangle - \langle\psi_0|H_0|\psi_0\rangle
    \end{equation}
    In fig. (\ref{fig:binding}) we report the binding energy curve, namely the value of the binding energy as a function of the interatomic distance $d$, for a fixed value of the angle $\theta$. We choose $\theta=0.58$ illustrating most of the results.

    \begin{figure}[H]
    \label{fig:binding}
        %\hspace{1.25cm}
        %\begin{subfigure}
        \includegraphics[scale=0.74]{figures/binding.pdf}
        %\end{subfigure}
        \caption{Binding energy curve for the model at angle $\theta=0.58$, together with its Morse fit.}
    \end{figure}

    On fig. (\ref{fig:binding}) we observe a perfect agreement with a fit of Morse type:
    \begin{equation}
    \label{eq:morse_fit}
        f(d) = E_b\left(e^{-2\frac{d-d_b}{s}} - 2e^{-\frac{d-d_b}{s}}\right)\,.
    \end{equation}
    The location of the bound state is given by $d_b\simeq 0.54$, with energy $-D_b\simeq 0.46$. The length scale $s$ is given by $s\simeq 2.75$. The quality of the Morse fit actully increases monotonically with the angle, cf. fig. (\ref{fig:morse_quality}), the quality being defined by the $\ell^2$-distance between the numerical result and the fit.

    \begin{figure}
    \label{fig:morse_quality}
        %\hspace{1.25cm}
        %\begin{subfigure}
        \includegraphics[scale=0.75]{figures/morse_quality.pdf}
        %\end{subfigure}
        \caption{\textbf{Blue curve:} Quality of the Morse fit as a function of the angle $\theta$, defined as the $\ell^2$-norm of the difference between the simulation and the Morse fit. \textbf{Yellow line:} Minimal angle above which the models are considered smooth. \textbf{Red line:} Maximal angle above which the models do not exhibit binding anymore.}
    \end{figure}

    By varying the value of the angle $\theta$, we observe two different regimes. For small values of the angle, the curve deviates more and more from a Morse curve, and in the extreme longitudinal case $\theta=0$, becomes highly non-smooth at short interatomic distances. This small angle regime is also characterized by the existence of a neagative global minimum of the binding energy, hence by the existence of a bound state. On the other hand, as the angle increases, the $\ell^2$-quality of the fit becomes better. However, passed some value of the angle, the global minimum disappears together with the corresponding bound state. This competition between smoothness and existence of a bound state is illustrated in fig. (\ref{fig:binding_vs_smooth}).

    \begin{figure}
    \label{fig:binding_vs_smooth}
        %\hspace{1.25cm}
        %\begin{subfigure}
        \includegraphics[scale=0.75]{figures/binding_vs_smooth.pdf}
        %\end{subfigure}
        \caption{Binding energy curve for three different values of the angle $\theta$ illustrating the tension between models close to $\theta=0$ and models close to $\theta=\pi/2$. For small angle (blue curve),  the strong curvature prevents a good Morse fit, while for large angle (green curve), the transverse configuration of the drudons prevents formation of a bound state. The orange curve corresponds an intermediate angle exhibiting both binding and an excellent Morse fit.}
    \end{figure}

    Both regimes can be understood physically as follows: for very small angle, and in particular for the longitudinal model $\theta=0$, as the two QDOs are getting closer and closer to each other, unstable configurations in which the two drudons are getting arbitrarily close to each other start appearing. Indeed, space being 1-dimensional and the two drudons moving along a common axis, as the two drudons get close to one another, the associated Coulom repulsion component of the ground state energy diverges. On the other hand, for large angle, and \textit{a fortiori} for the transverse model $\theta=\pi/2$, two main configurations of the drudons may occur (thinking classically) depending on the relative position of the drudons with respect to the axis connecting the two nuclei. In the first configuration, the drudons are sitting on opposite sides. In that case, the dominant contribution to the energy between the two QDOs is the Coulomb repulsion between the nuclei. In the second configuration, the drudons are on the same side. In that case, in adding to the repulsive force between the two nuclei, one can also add up the repulsion force between the two drudons, leading to an even more repulsive scenario. Summing up, no binding can occur at $\theta=\pi/2$, and by smoothness of the binding energy as a function of $\theta$, this should also be the case in an open neighborhood of $\theta=\pi/2$.

    The above observation suggests the following recipe. The longitudinal model predict the existence of negative minima of the binding energy curve. It is however unstable due to the configurations of superposed drudons, as explained above. One can then \textit{regularize} this 1d model by allowing for a non-zero angle $\theta$, and slightly increase it until reaching a certain level of smoothness, that we have chosen here to be quantified by the quality of a Morse fit. The angle should however not be too large, smaller than the transition point beyond which the bound state disappears. This procedure defines a small range of models characterized by an angle $\theta\in[\theta_\text{min}, \theta_\text{max}]$, as illustrated on fig. (\ref{fig:morse_quality}). Finally, the QDO parameters are tuned to match \textit{ab initio} computations for the two-body system of interest.

\subsection{Ground state wavefunction and entanglement entropy}

    \begin{figure*}
    \label{fig:wigners_joint}
        %\hspace{1.25cm}
        %\begin{subfigure}
        \includegraphics[scale=0.75]{figures/wigners_joint.pdf}
        %\end{subfigure}
        \caption{\textbf{From top to bottom:} $d=3.16$: QDOs are far apart, $d=1.36$: QDOs start feeling each other, $d=0.82$: entanglement entropy is maximal, $d=0.54$: deep in the bound state. \textbf{From left to right:} Wigner distribution of the left QDO, Wigner distribution of the right QDO, joint position quadrature distribution of the two drudons.}
    \end{figure*}

    Let us give the expression of the partial densitity matrix associated to $\text{QDO}_1$. The total state of the system, we recall, is expressed in the Fock basis as the pure state:
    \begin{equation}
        |\psi\rangle = \sum_{n_1,n_{2}=0}^\infty \alpha_{n_1n_2}|n_1\rangle\otimes|n_2\rangle\,.
    \end{equation}
    This state is directly accessible when using the simulator, and can be obtained on a genuine hardware by state tomography techniques \cite{Lvovsky:2009zz}. Given the state, the density matrix of the system is simply given by:
    \begin{equation*}
        \rho = \sum_{\substack{n_1,n_2 \\ m_1,m_2}} \alpha^*_{m_1m_2}\alpha_{n_1n_2}|n_1\rangle\langle m_1|\otimes|n_2\rangle\langle m_2|\,.
    \end{equation*}
    The partial trace associated to $\text{QDO}_1$ is therefore given by:
    \begin{equation}
        \rho_1 = \sum_{n, m, l} \alpha^*_{ml}\alpha_{nl}\,|n\rangle\langle m|\,.
    \end{equation}
    Since the state of the total system is pure, the von Neumann entropy of the total density matrix is zero. $\text{QDO}_2$ can then be interpreted as purifying the system composed solely of $\text{QDO}_1$. The two QDOs therefore have identical von Neumann entropy $S(\rho_1)$, the entanglement entropy. The quantum mutual information of the system is therefore given by
    \begin{equation}
        I(1:2) = S(\rho_1) + S(\rho_2) - S(\rho) = 2S(\rho_1) \,,
    \end{equation}
    with the von Neumann entropy being defined as
    \begin{equation}
        S(\rho) = -\text{Tr}\left[\rho\log\rho\right]\,.
    \end{equation}
    Another intersting quantity to consider is the quantum correlation between the position quadrature of the two QDOs. Denoting again by angular brackets the expectation of an observable in the ground state, the correlation coefficient is defined by:
    \begin{equation}
        C(X_1, X_2) = \frac{\langle X_1X_2\rangle - \langle X_1\rangle\langle X_2\rangle}{\sqrt{\langle X_1^2\rangle - \langle X_1\rangle^2}\sqrt{\langle X_2^2\rangle - \langle X_2\rangle^2}}\,.
    \end{equation}
    The profile of the quantum mutual information as a function of the interatomic distance is provided in fig. (\ref{fig:entropy_correlation}).

    \begin{figure}
    \label{fig:entropy_correlation}
        %\hspace{1.25cm}
        %\begin{subfigure}
        \includegraphics[scale=0.9]{figures/entropy_correlation.pdf}
        %\end{subfigure}
        \caption{\textbf{On the top.} Entanglement entropy vs. interatomic distance. \textbf{On the bottom.} Position quadratures correlation coefficient vs. interatomic distance. Both curves correspond to the model at angle $\theta=0.58$.}
    \end{figure}



    \begin{figure}
    \label{fig:second_derivatives}
        %\hspace{1.25cm}
        %\begin{subfigure}
        \includegraphics[scale=0.755]{figures/repulsion_vs_correlation_2nd_derivative.pdf}
        %\end{subfigure}
        \caption{Competition between correlation and Coulomb repulsion energies for the model at $\theta=0.58$. Their second derivatives intersect at the inflexion point of the binding curve.}
    \end{figure}




    Note that the maximum of the entanglement entropy is attained around the inflexion point of the the binding energy curve. Indeed, the binding energy is composed of a Coulomb repulsion and an attractive correlation energy contribution:
    \begin{equation}
        E(d)=E_\text{Rep}(d)+E_\text{Corr}(d)
    \end{equation}
    The inflexion point $d_\star$, which solves
    \begin{equation}
        \frac{\partial^2 E_\text{Rep}}{\partial d^2}(d_\star) = -\frac{\partial^2 E_\text{Corr}}{\partial d^2}(d_\star)\,,
    \end{equation}
    can be interpreted as the transition between a large distance regime in which it is beneficial for the system to increase correlation in order to lower its ground state energy, and a short range regime in which the Coulomb repulsion becomes dominant. In terms of the Morse fit (\ref{eq:morse_fit}), this point is given by
    \begin{equation}
        d_\star = d_b + \log(2)s
    \end{equation}





\section{Conclusion}

    In this letter, we showed that continuous variables quantum photonics quantum computing is particularly adapted to the study of the full-Coulomb Many Body Dispersion model, the fundamental degrees of freedom of the latter being bosonic in nature. Beyond standing as a proof-of-concept that NISQ algorithms can be successfully applied to quantum chemistry problems beyond the usual approach using the second quantized formulation of the electronic structure problem for small molecules, we observed that the standard QDO toy model for Many Body Dispersion can be further simplified by reducing it to a single effective spatial dimension (at least for the two-QDO system), together with a reparameterization of the QDO parameters.

\begin{acknowledgments}

    We would like to thank Dahvyd Wing and Kyunghoon Han for important discussions.

\end{acknowledgments}

\section*{Code Availability}

The reader will find an open source python code accompanying this paper at \href{https://github.com/MatthieuSarkis/qdo}{github repository}.

\appendix

\nocite{*}
\bibliographystyle{IEEEtran}
\bibliography{bibliography}

\end{document}


%%%%%%%%%%%%%%%%%%%%%%%%%%%%%%%%%%%%%%%%%%%%%%%%%%%%
%%%%%%%%%%% UNUSED CODE %%%%%%%%%%%%%%%%%%%%%%%%%%%%
%%%%%%%%%%%%%%%%%%%%%%%%%%%%%%%%%%%%%%%%%%%%%%%%%%%%


Let us give here the expression for the multipolar potential up to quartic order:
        \begin{align}
            V_0(\bm{x} _i, \bm{x} _j) &= q_iq_j\frac{x_ix_j + y_iy_j - z_iz_j}{r_{ij}^3}\\
            V_1(\bm{x} _i, \bm{x} _j) &= \frac{q_iq_j}{2r_{ij}^4}\big(-3  x_i ^2  z_j -6  x_i   x_j   z_i +6  x_i   x_j   z_j +3  x_j ^2  z_i\nonumber\\
            & -3  y_i ^2  z_j -6  y_i   y_j   z_i +6  y_i   y_j   z_j +3  y_j ^2  z_i\nonumber\\
            & +6  z_i ^2  z_j -6  z_i   z_j ^2\big)\\
            V_2(\bm{x} _i, \bm{x} _j) &= \frac{q_iq_j}{2 r_{ij}^4}\big(-6  x_i ^3  x_j +9  x_i ^2  x_j ^2-6  x_i ^2  y_i   y_j +3  x_i ^2  y_j ^2\nonumber\\
            &+24  x_i ^2  z_i   z_j -12  x_i ^2  z_j ^2-6  x_i   x_j ^3-6  x_i   x_j   y_i ^2\nonumber\\
            &+12  x_i   x_j   y_i   y_j -6  x_i   x_j   y_j ^2+24  x_i   x_j   z_i ^2\nonumber\\
            &-48  x_i   x_j   z_i   z_j +24  x_i   x_j   z_j ^2+3  x_j ^2  y_i ^2-6  x_j ^2  y_i   y_j \nonumber\\
            &-12  x_j ^2  z_i ^2+24  x_j ^2  z_i   z_j -6  y_i ^3  y_j +9  y_i ^2  y_j ^2\nonumber\\
            &+24  y_i ^2  z_i   z_j -12  y_i ^2  z_j ^2-6  y_i   y_j ^3+24  y_i   y_j   z_i ^2\nonumber\\
            &-48  y_i   y_j   z_i   z_j +24  y_i   y_j   z_j ^2-12  y_j ^2  z_i ^2+24  y_j ^2  z_i   z_j \nonumber\\
            &-16  z_i ^3  z_j +24  z_i ^2  z_j ^2-16  z_i   z_j ^3\big)
        \end{align}



        \begin{equation}
            \begin{split}
                &V_0(x_i, x_j) = q_iq_j\frac{x_ix_j}{r_{ij}^3}\times\\
                &\times\begin{cases}
                    1-3\cos^2\theta, & \text{generic case} \\
                    -2, & \text{parallel case} \\
                    1, & \text{perpendicular case} \\
                    -\frac{1}{2}, & \text{oblique case} \\
                    -2-6\epsilon, & \text{regularized case}
                \end{cases}
            \end{split}
            \end{equation}
            The next terms in the multipolar expansion are:
            \begin{equation}
            \begin{split}
                &V_1(x_i, x_j) = q_iq_j\frac{x_ix_j(x_i-x_j)}{r_{ij}^4}\times\\
                &\times\begin{cases}
                    \frac{3\cos\theta(-3+5\cos^2\theta)}{2}, & \text{generic case} \\
                    3, & \text{parallel case} \\
                    0, & \text{perpendicular case} \\
                    -\frac{3}{4\sqrt 2}, & \text{oblique case} \\
                    3+18\epsilon, & \text{regularized case}
                \end{cases}
            \end{split}
            \end{equation}
            \begin{equation}
                \begin{split}
                    &V_2(x_i, x_j) = q_iq_j\frac{x_ix_j(2x_i^2-3x_ix_j+2x_j^2)}{r_{ij}^5}\times\\
                    &\times
                    \begin{cases}
                        -\frac{3-30\cos^2\theta+35\cos^4\theta}{4}, & \text{generic case} \\
                        -2, & \text{parallel case} \\
                        -\frac{3}{4}, & \text{perpendicular case} \\
                        \frac{13}{16}, & \text{oblique case} \\
                        -2-20\epsilon, & \text{regularized case}
                    \end{cases}
                \end{split}
            \end{equation}

            \subsection{12 different models}

            For the simulation, we will restrict the system to two QDOs. Depending on the space dimensionality (1d oblique, 1d parallel, 1d perpendicular, 3d) and the choice of potential (quadratic, quartic, Coulomb), we therefore have 12 models to try and study, gathered in table (\ref{tab:models}).
            \begin{table}[ht!]
            \caption{\label{tab:models} The twelve models}
            \begin{ruledtabular}
            \begin{tabular}{c|cccc}
                \diagbox[height=1.8\line]{\textsc{pot}}{\textsc{dim}}& 1d oblique & 1d parallel & 1d perpendicular & 3d \\
                \hline\\[-0.95em]
                \textsc{quad} & $H_{(1,0)}$ & $H_{(1,1)}$ & $H_{(1,2)}$ & $H_{(1,3)}$ \\
                \textsc{quart} & $H_{(2,0)}$ & $H_{(2,1)}$ & $H_{(2,2)}$ & $H_{(2,3)}$\\
                \textsc{Coulomb} & $H_{(3,0)}$ & $H_{(3,1)}$ & $H_{(3,2)}$ & $H$ \\
            \end{tabular}
            \end{ruledtabular}
            \end{table}

            The space regularized model Coulomb potential reads
            \begin{equation}
            \begin{split}
                &\frac{V_\text{Coul}^\epsilon(x_i, x_j)}{q_iq_j} = \frac{1}{r_{ij}} - \frac{1}{\sqrt{r_{ij}^2+2r_{ij}(1+\epsilon) x_i+x_i^2}} \\
                &- \frac{1}{\sqrt{r_{ij}^2-2r_{ij}(1+\epsilon) x_j+x_j^2}} \\
                &+\frac{1}{\sqrt{r_{ij}^2 - 2r_{ij}(1+\epsilon) (x_j-x_i) + (x_j-x_i)^2}}\,.
            \end{split}
            \end{equation}

            In terms of components, the full Coulomb potential reads:
            \begin{equation}
            \mathclap{
            \begin{split}
                &\frac{V_\text{Coul}(\bm{x} _i, \bm{x} _j)}{q_iq_j} = \frac{1}{r_{ij}} - \frac{1}{\sqrt{r_{ij}^2 + x_i^2+y_i^2+z_i^2+2rz_i}} \\
                &- \frac{1}{\sqrt{r_{ij}^2 + x_j^2+y_j^2+z_j^2-2r_{ij}z_j}} \\
                &+\frac{1}{\sqrt{r_{ij}^2 + (x_j-x_i)^2+(y_j-y_i)^2+(z_j-z_i)^2-2r_{ij}(z_j-z_i)}}
            \end{split}
            }
            \end{equation}

            For the full Coulomb potential, assuming that the drudons are constrained to move along an axis, we get the following expressions:
            \begin{equation}
            \begin{split}
                \frac{V_\text{Coul}^\perp(x_1, x_2)}{q_1q_2} = &\ \frac{1}{r_{12}} - \frac{1}{\sqrt{r_{12}^2+x_1^2}} - \frac{1}{\sqrt{r_{12}^2+x_2^2}} \\
                &+\frac{1}{\sqrt{r_{12}^2 + (x_2-x_1)^2}}
            \end{split}
            \end{equation}
            in the case where the drudons move perpendicular to the axis joining the two nuclei, and
            \begin{equation*}
            \mathclap{
                \frac{V_\text{Coul}^{||}(x_1, x_2)}{q_1q_2} = \frac{1}{r_{12}} - \frac{1}{|r_{12}+x_1|} - \frac{1}{|r_{12}-x_2|} + \frac{1}{|r_{12}+x_1-x_2|}
            }
            \end{equation*}
            in the case where they move parallel to the latter.


            \begin{equation}
                |\psi(\omega)\rangle = \sum_{n_1,\dots,n_{2K}=0}^\infty \alpha_{n_1\dots n_{2K}}|n_1\rangle\otimes\dots\otimes|n_{2K}\rangle\,.
            \end{equation}
            The modes labeled by $(n_1,\dots,n_K)$ correspond to $\text{QDO}_1$, while the modes $(n_{K+1},\dots,n_{2K})$ are attached to $\text{QDO}_2$.
            The amplitude of a specific tuple of the quadratures $(X_1,\dots, X_{K})$ is therefore given by:
            \begin{equation*}
            \mathclap{
                \langle X_1,\dots,X_{2K}|\psi\rangle = \sum_{n_1,\dots,n_{2K}=0}^\infty \alpha_{n_1\dots n_{2K}}\prod_{i=1}^{2K}\frac{e^{-\sum_{i=1}^{2K}\frac{X_i^2}{2}}H_{n_i}(X_i)}{\sqrt{\pi^{1/2}2^{n_i}n_i!}}\,,
            }
            \end{equation*}
            in terms of the Hermite polynomials. The joint law of the quadratures in the state $|\psi\rangle$ is therefore given by
            \begin{equation}
            \begin{split}
                \rho(X_1,\dots,X_{2K}) &= \sum_{\substack{n_1,\dots,n_{2K} \\ m_1,\dots,m_{2K}}} \alpha_{n_1\dots n_{2K}}\alpha^*_{m_1\dots m_{2K}}\\
                &\times\prod_{i=1}^{2K}\frac{e^{-X_i^2}H_{n_i}(X_i)H_{m_i}(X_i)}{\sqrt{\pi^{1/2}2^{n_i}n_i!}\sqrt{\pi^{1/2}2^{m_i}m_i!}}
            \end{split}
            \end{equation}
            After extracting as well the mean photon numbers $\langle n_{i,\alpha}\rangle$, one obtains $\langle H\rangle$.

            \begin{algorithm}
                \caption{Computation of the loss}\label{alg:loss_computation}
                    \textbf{Parameters:} $M\in\mathbb N$

                    \KwResult{Value of the loss $\mathcal C$}\

                    Initialize $\mathcal C \gets 0$\;
                    Get the position quadratures distribution with alg. (\ref{alg:statistics_computation})\;
                    Get the photon numbers distribution with alg. (\ref{alg:statistics_computation})\;
                    Compute the loss $\mathcal C$ using eq. (\ref{eq:loss})\;
                    \textbf{return} $\mathcal C$.
            \end{algorithm}

            \begin{equation}
                |\psi\rangle = \sum_{n_1,\dots,n_{2K}=0}^\infty \alpha_{n_1\dots n_{2K}}|n_1\rangle\otimes\dots\otimes|n_{2K}\rangle\,,
            \end{equation}
            leading to the following expression for the density matrix:
            \begin{equation*}
            \mathclap{
                \rho = \sum_{\substack{n_1,\dots,n_{2K} \\ m_1,\dots,m_{2K}}} \alpha^*_{m_1\dots m_{2K}}\alpha_{n_1\dots n_{2K}}|n_1\rangle\langle m_1|\otimes\dots\otimes|n_{2K}\rangle\langle m_{2K}|\,.
            }
            \end{equation*}
            The partial trace associated to $\text{QDO}_1$ is therefore given by:
            \begin{equation}
            \begin{split}
                \rho_1 = \sum_{\substack{n_1,\dots,n_{K} \\ m_1,\dots,m_{K} \\ l_1,\dots,l_{K}}}& \alpha^*_{m_1\dots m_{K}l_1,\dots,l_{K}}\alpha_{n_1\dots n_{K}l_1,\dots,l_{K}}\\
                &|n_1\rangle\langle m_1|\otimes\dots\otimes|n_{K}\rangle\langle m_{K}|\,.
            \end{split}
            \end{equation}