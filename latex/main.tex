\documentclass[reprint, amsmath, amssymb, aps]{revtex4-2}

\usepackage{graphicx}
\graphicspath{{./assets/figures/}}
\usepackage{dcolumn}
\usepackage{bm}
\usepackage{diagbox}
\usepackage[table]{xcolor}
\usepackage{hyperref}
\usepackage{comment}
\newcolumntype{L}{>{$}l<{$}} % math-mode version of "l" column type
\usepackage{float} % To fix location of tables with H
\usepackage[ruled,vlined]{algorithm2e}
\usepackage{braket}
\usepackage{amsthm}
\usepackage{cleveref}
\usepackage{mathtools}
\usepackage{listings} % to add python code

\newtheorem{definition}{Definition}
\newtheorem{prop}{Proposition}
\DeclareMathOperator{\tr}{Tr}

\begin{document}

\preprint{}

\title{Simulating Quantum Drude Oscillators on a photonic quantum computer}
%\thanks{}

\author{Matthieu Sarkis}
\email{matthieu.sarkis@uni.lu}
\affiliation{
Department of Physics and Materials Science\\ University of Luxembourg, L-1511, Luxembourg City, Luxembourg.
}

\author{}
\email{}
\affiliation{
}

%\collaboration{}%\noaffiliation

\date{\today}

\begin{abstract}
%\begin{description}
%\item[Usage]
%Secondary publications and information retrieval purposes.
%\item[Structure]
%You may use the \texttt{description} environment to structure your abstract;
%use the optional argument of the \verb+\item+ command to give the category of each item.
%\end{description}
\end{abstract}

%\keywords{Suggested keywords}%Use showkeys class option if keyword
                              %display desired
\maketitle

%\tableofcontents


\section{Introduction}

    We work in atomic units, for which $m_e=q_e=\hbar=\frac{1}{4\pi\epsilon_0}=1$.

\section{Definition of the model}

    \subsection{Three-dimensional model}

        The Hamiltonian describing a system of $N$ QDOs in 3d is given by:
        \begin{equation}
        \label{eq:full_QDO_Hamiltonian}
            H_{(3,3)}=\sum_{i=1}^N\left[\frac{\bm{p} _i^2}{2m_i} + \frac{1}{2}m_i\omega_i^2\bm{x} _i^2\right] +\sum_{i<j}V_\text{Coul}\left(\bm{x} _i, \bm{x} _j\right)\,,
        \end{equation}
        with the Coulomb interaction receiving contributions from every pair of constituents (centers and point particles):
        \begin{equation*}
        \label{eq:full_coulomb_potential}
        \mathclap{
            \frac{V_\text{Coul}\left(\bm{x} _i, \bm{x} _j\right)}{q_iq_j}=\frac{1}{r_{ij}} - \frac{1}{|\bm{r}_{ij}  + \bm{x} _i|} - \frac{1}{|\bm{r}_{ij}  - \bm{x} _j|} + \frac{1}{|\bm{r}_{ij} - \bm{x} _j + \bm{x} _i|}
        }
        \end{equation*}
        The subscript (3,3) in the above definition will become clear soon.
        In terms of components, the full Coulomb potential reads:
        \begin{equation}
        \mathclap{
        \begin{split}
            &\frac{V_\text{Coul}(\bm{x} _i, \bm{x} _j)}{q_iq_j} = \frac{1}{r_{ij}} - \frac{1}{\sqrt{r_{ij}^2 + x_i^2+y_i^2+z_i^2+2rz_i}} \\
            &- \frac{1}{\sqrt{r_{ij}^2 + x_j^2+y_j^2+z_j^2-2r_{ij}z_j}} \\
            &+\frac{1}{\sqrt{r_{ij}^2 + (x_j-x_i)^2+(y_j-y_i)^2+(z_j-z_i)^2-2r_{ij}(z_j-z_i)}}
        \end{split}
        }
        \end{equation}
        In the multipolar expansion, this can be expressed as a power series in the inverse distance separating the two centers:
        \begin{equation}
            V_\text{Coul}\left(\bm{x} _i, \bm{x} _j\right)= \sum_{n\geq 0} V_n\left(\bm{x} _i, \bm{x} _j\right)\,,
        \end{equation}
        with the following scaling behavior in terms of the distance between the centers:
        \begin{equation}
            V_n\left(\bm{x} _i, \bm{x} _j\right)\propto r_{ij}^{-n-3}\,.
        \end{equation}
        The potential $V_0$ corresponds then to the dipole-dipole interaction, $V_1$ to the dipole-quadrupole interaction, and $V_2$ to the quadrupole-quadrupole and dipole-octupole interaction.
        Let us give here the expression for the multipolar potential up to quartic order:
        \begin{align}
            V_0(\bm{x} _i, \bm{x} _j) &= q_iq_j\frac{x_ix_j + y_iy_j - z_iz_j}{r_{ij}^3}\\
            V_1(\bm{x} _i, \bm{x} _j) &= \frac{q_iq_j}{2r_{ij}^4}\big(-3  x_i ^2  z_j -6  x_i   x_j   z_i +6  x_i   x_j   z_j +3  x_j ^2  z_i\nonumber\\
            & -3  y_i ^2  z_j -6  y_i   y_j   z_i +6  y_i   y_j   z_j +3  y_j ^2  z_i\nonumber\\
            & +6  z_i ^2  z_j -6  z_i   z_j ^2\big)\\
            V_2(\bm{x} _i, \bm{x} _j) &= \frac{q_iq_j}{2 r_{ij}^4}\big(-6  x_i ^3  x_j +9  x_i ^2  x_j ^2-6  x_i ^2  y_i   y_j +3  x_i ^2  y_j ^2\nonumber\\
            &+24  x_i ^2  z_i   z_j -12  x_i ^2  z_j ^2-6  x_i   x_j ^3-6  x_i   x_j   y_i ^2\nonumber\\
            &+12  x_i   x_j   y_i   y_j -6  x_i   x_j   y_j ^2+24  x_i   x_j   z_i ^2\nonumber\\
            &-48  x_i   x_j   z_i   z_j +24  x_i   x_j   z_j ^2+3  x_j ^2  y_i ^2-6  x_j ^2  y_i   y_j \nonumber\\
            &-12  x_j ^2  z_i ^2+24  x_j ^2  z_i   z_j -6  y_i ^3  y_j +9  y_i ^2  y_j ^2\nonumber\\
            &+24  y_i ^2  z_i   z_j -12  y_i ^2  z_j ^2-6  y_i   y_j ^3+24  y_i   y_j   z_i ^2\nonumber\\
            &-48  y_i   y_j   z_i   z_j +24  y_i   y_j   z_j ^2-12  y_j ^2  z_i ^2+24  y_j ^2  z_i   z_j \nonumber\\
            &-16  z_i ^3  z_j +24  z_i ^2  z_j ^2-16  z_i   z_j ^3\big)
        \end{align}
        We define the dimensionless position and momenta associated to QDO $i$:
        \begin{equation}
            \bm{X}_i := \sqrt{\frac{m_i\omega_i}{\hbar}}\,\bm{x}_i\,,\ \ \ \ \ \bm{P}_i := \frac{\bm{p}_i}{\sqrt{2\hbar m_i\omega_i}}\,,
        \end{equation}
        in terms of which the $3N$ creation and annihilation operators read ($\alpha=1,2,3$)
        \begin{equation}
            \bm a_{i} = \frac{\bm X_{i} + i\bm P_{i}}{\sqrt 2}\,,\ \ \ \ \ \bm a^\dagger_{i} = \frac{\bm X_{i} - i\bm P_{i}}{\sqrt 2}\,.
        \end{equation}
        In terms of the dimensionless canonical variables, the Hamiltonian reads
        \begin{equation}
        \begin{split}
            H_{(3,3)} &= \sum_{i=1}^N\frac{\hbar\omega_i}{2}\left(\bm X_{i}^2 + \bm P_{i}^2\right) \\
            & + \sum_{i<j}V_\text{Coul}\left(\sqrt{\frac{\hbar}{m_i\omega_i}}\bm{X} _i, \sqrt{\frac{\hbar}{m_j\omega_j}}\bm{X} _j\right)
        \end{split}
        \end{equation}
        and can be rewritten
        \begin{equation}
        \begin{split}
            H_{(3,3)} &= \sum_{i=1}^N\frac{\hbar\omega_i}{2}\left(\bm a_{i}^\dagger\cdot\bm a_{i} +\frac{3}{2}\right) \\
            & + \sum_{i<j}V_\text{Coul}\left(\sqrt{\frac{\hbar}{m_i\omega_i}}\frac{\bm a_i + \bm a_i^\dagger}{\sqrt 2}, \sqrt{\frac{\hbar}{m_j\omega_j}}\frac{\bm a_j + \bm a_j^\dagger}{\sqrt 2}\right)
        \end{split}
        \end{equation}
    \subsection{One-dimensional case}

        We consider the one dimensional system in which the electrons are constrained to move either in the direction parallel to the axis separating the two nuclei, or perpendicular to the latter.
        Those two cases can be obtained simply by setting to zero the contribution from all the terms involving the $(x,y)$ Cartesian axes in the parallel case, and the $(y,z)$ Cartesian axes in the perpendicular case. In both cases, we denote by $x$ the remaining degree of freedom.
        \begin{equation}
        \begin{split}
            &V_0(x_i, x_j) = q_iq_j\frac{x_ix_j}{r_{ij}^3}\times\\
            &\times\begin{cases}
                1-3\cos^2\theta, & \text{generic case} \\
                -2, & \text{parallel case} \\
                1, & \text{perpendicular case} \\
                -\frac{1}{2}, & \text{oblique case}
            \end{cases}
        \end{split}
        \end{equation}
        The next terms in the multipolar expansion are:
        \begin{equation}
        \begin{split}
            &V_1(x_i, x_j) = q_iq_j\frac{x_ix_j(x_i-x_j)}{r_{ij}^4}\times\\
            &\times\begin{cases}
                \frac{3\cos\theta(-3+5\cos^2\theta)}{2}, & \text{generic case} \\
                3, & \text{parallel case} \\
                0, & \text{perpendicular case} \\
                -\frac{3}{4\sqrt 2}, & \text{oblique case}
            \end{cases}
        \end{split}
        \end{equation}
        \begin{equation}
            \begin{split}
                &V_2(x_i, x_j) = q_iq_j\frac{x_ix_j(2x_i^2-3x_ix_j+2x_j^2)}{r_{ij}^5}\times\\
                &\times
                \begin{cases}
                    -\frac{3-30\cos^2\theta+35\cos^4\theta}{4}, & \text{generic case} \\
                    -2, & \text{parallel case} \\
                    -\frac{3}{4}, & \text{perpendicular case} \\
                    \frac{13}{16}, & \text{oblique case}
                \end{cases}
            \end{split}
        \end{equation}
        For the full Coulomb potential, assuming that the electrons are constrained to move along an axis, we get the following expressions:
        \begin{equation}
        \begin{split}
            \frac{V_\text{Coul}^\perp(x_i, x_j)}{q_iq_j} = &\ \frac{1}{r_{ij}} - \frac{1}{\sqrt{r_{ij}^2+x_i^2}} - \frac{1}{\sqrt{r_{ij}^2+x_j^2}} \\
            &+\frac{1}{\sqrt{r_{ij}^2 + (x_j-x_i)^2}}
        \end{split}
        \end{equation}
        in the case where the electrons move perpendicular to the axis joining the two nuclei, and
        \begin{equation*}
        \mathclap{
            \frac{V_\text{Coul}^{||}(x_i, x_j)}{q_iq_j} = \frac{1}{r_{ij}} - \frac{1}{|r_{ij}+x_i|} - \frac{1}{|r_{ij}-x_j|} + \frac{1}{|r_{ij}+x_i-x_j|}
        }
        \end{equation*}
        in the case where they move parallel to the latter. In the case of a generic angle $\theta$ between $\bm r_{ij}$ and $\bm x_i$/$\bm x_j$, one has
        \begin{equation}
            \begin{split}
                &\frac{V_\text{Coul}^\theta(x_i, x_j)}{q_iq_j} = \frac{1}{r_{ij}} - \frac{1}{\sqrt{r_{ij}^2+2r_{ij}(\cos\theta) x_i+x_i^2}} \\
                &- \frac{1}{\sqrt{r_{ij}^2-2r_{ij}(\cos\theta) x_j+x_j^2}} \\
                &+\frac{1}{\sqrt{r_{ij}^2 - 2r_{ij}(\cos\theta) (x_j-x_i) + (x_j-x_i)^2}}\,.
            \end{split}
            \end{equation}

    \subsection{12 different models}

        For the simulation, we will restrict the system to two QDOs. Depending on the space dimensionality (1d oblique, 1d parallel, 1d perpendicular, 3d) and the choice of potential (quadratic, quartic, Coulomb), we therefore have 12 models to try and study, gathered in table (\ref{tab:models}).
        \begin{table}[ht!]
        \caption{\label{tab:models} The twelve models}
        \begin{ruledtabular}
        \begin{tabular}{c|cccc}
            \diagbox[height=1.8\line]{\textsc{pot}}{\textsc{dim}}& 1d oblique & 1d parallel & 1d perpendicular & 3d \\
            \hline\\[-0.95em]
            \textsc{quad} & $H_{(1,0)}$ & $H_{(1,1)}$ & $H_{(1,2)}$ & $H_{(1,3)}$ \\
            \textsc{quart} & $H_{(2,0)}$ & $H_{(2,1)}$ & $H_{(2,2)}$ & $H_{(2,3)}$\\
            \textsc{Coulomb} & $H_{(3,0)}$ & $H_{(3,1)}$ & $H_{(3,2)}$ & $H_{(3,3)}$ \\
        \end{tabular}
        \end{ruledtabular}
        \end{table}

\section{Photonic circuit}

    The circuit implements a unitary $U(\theta)$ acting on an input reference state (the Fock vacuum for instance) that we simply take to be the vacuum state $|0\rangle$. The state prepared by the circuit is therefore given by
    \begin{equation}
        |\psi(\theta)\rangle = U(\theta)|0\rangle\,.
    \end{equation}

    In the dipolar approximation, namely for the Hamiltonians $H_{(1,1)}$, $H_{(1,2)}$ and $H_{(1,3)}$, we expect that using a Gaussian state would be enough. The circuit is therefore composed of at most quadratic optical components (squeezing operations for our ansatz). However for the other models,  non-Gaussian operation should be added in the end of each layer in the ansatz circuit.

    One the ansatz state $|\psi(\theta)\rangle$ has been produced, one should extract the value of the energy in that state, namely compute the value of
    \begin{equation}
        \langle\psi(\theta)|H|\psi(\theta)\rangle
    \end{equation}
    To be specific, let us take the model $H_{(3,3)}$, and let us denote by angular brackets the expectation in state $|\psi(\theta)\rangle$. One has
    \begin{equation}
    \label{eq:loss}
    \begin{split}
        &\langle H_{(3,3)}\rangle = \sum_{i=1}^N\frac{\hbar\omega_i}{2}\left(\langle n_{i,x}\rangle+\langle n_{i,y}\rangle+\langle n_{i,z}\rangle+\frac{3}{2}\right)\\
        &+ \sum_{i<j}\left\langle V_\text{Coul}\left(\sqrt{\frac{\hbar}{m_i\omega_i}}\bm X_i, \sqrt{\frac{\hbar}{m_j\omega_j}}\bm X_j\right)\right\rangle
    \end{split}
    \end{equation}
    by linearity of the expectation. On the second line one has to compute something of the form $\langle f(X_{i,\alpha})\rangle$, where $(X_{i,\alpha})$ denotes collectively the position quadrature of all the photon channels (including QDO and spatial index). One therefore needs to extract the statistics of quadratures by preparing and measuring the state $|\psi(\theta)\rangle$ in the quadrature basis. Once the joint density $\rho$ of $(X_{i,\alpha})$ in the state $|\psi(\theta)\rangle$ is known, one can compute
    \begin{equation}
        \langle f(X_{i,\alpha})\rangle = \int_{\mathbb R^{3N}}f(x_{i,\alpha})\rho(x_{i,\alpha})\prod_{i,\alpha}\text{d}x_{i,\alpha}\,.
    \end{equation}
    There is a complication related to the fact that strawberryfields does not allow to access the joint law of the position quadratures. We therefore have to extract it by ourselves. Let us first suppose that there are $K$ photon modes. The statevector is represented in the Fock basis as follows:
    \begin{equation}
        |\psi\rangle = \sum_{n_1,\dots,n_{K}=0}^\infty \alpha_{n_1\dots n_{K}}|n_1\rangle\otimes\dots\otimes|n_{K}\rangle\,.
    \end{equation}
    The amplitude of a specific tuple of the quadratures $(X_1,\dots, X_{K})$ is therefore given by:
    \begin{equation*}
    \mathclap{
        \langle X_1,\dots,X_{K}|\psi\rangle = \sum_{n_1,\dots,n_{K}=0}^\infty \alpha_{n_1\dots n_{K}}\prod_{i=1}^{K}\frac{e^{-\sum_{i=1}^{K}\frac{X_i^2}{2}}H_{n_i}(X_i)}{\sqrt{\pi^{1/2}2^{n_i}n_i!}}\,,
    }
    \end{equation*}
    in terms of the Hermite polynomials. The joint law of the quadratures in the state $|\psi\rangle$ is therefore given by
    \begin{equation}
    \begin{split}
        \rho(X_1,\dots,X_{K}) &= \sum_{\substack{n_1,\dots,n_{K} \\ m_1,\dots,m_{K}}} \alpha_{n_1\dots n_{K}}\alpha^*_{m_1\dots m_{K}}\\
        &\times\prod_{i=1}^{K}\frac{e^{-X_i^2}H_{n_i}(X_i)H_{m_i}(X_i)}{\sqrt{\pi^{1/2}2^{n_i}n_i!}\sqrt{\pi^{1/2}2^{m_i}m_i!}}
    \end{split}
    \end{equation}
    Notice that the tensor $\alpha_{n_1,\dots,n_{K}}$ is precisely the output of `state.ket()` in strawberryfields.
    Also the Hermite polynomials are implemented in `scipy.special.hermite`:
    \begin{lstlisting}[language=Python]
    from scipy import special
    p = special.hermite(3, monic=False)
    \end{lstlisting}

    After extracting as well the mean photon numbers $\langle n_{i,\alpha}\rangle$, one obtains $\langle H_{(3,3)}\rangle$.

\section{Variational algorithm}

    Depending on the model $H_{(\mu,\nu)}$ of interest, we define the following loss function:
    \begin{equation}
        \mathcal C_{(\mu,\nu)}(\theta) := \langle\psi(\theta)|H_{(\mu,\nu)}|\psi(\theta)\rangle
    \end{equation}
    In order to compute this loss, one therefore has to measure both the photon number operator on each channel, as well as the position quadrature operator on each channel, as described in the previous section.
    \newpage

    \begin{algorithm}
        \caption{Extract distribution of observable}\label{alg:statistics_computation}
            \textbf{Parameters:} statevector $|\psi\rangle$, observable $\mathcal O$, shots $M\in\mathbb N$

            \KwResult{Probability distribution of $\mathcal O$ in state $|\psi\rangle$}\

            \For{$m=1$ to $\dim(\text{spec}(\mathcal O))$}{
                Initialize $N_m \gets 0$;
            }
            \For{$j=1$ to $M$}{
            Measure the state $|\psi$ in the basis $\mathcal O$, obtain the eigenvalue $o_m$, set $N_m\gets N_m+1$\;
            }
            \For{$m=1$ to $\dim(\text{spec}(\mathcal O))$}{
                Normalize $N_m \gets N_m / M$;
            }
            \textbf{return} $(N_m)_m$.
    \end{algorithm}

    \begin{algorithm}
        \caption{Computation of the loss}\label{alg:loss_computation}
            \textbf{Parameters:} $M\in\mathbb N$

            \KwResult{Value of the loss $\mathcal C$}\

            Initialize $\mathcal C \gets 0$\;
            Get the position quadratures distribution with alg. (\ref{alg:statistics_computation})\;
            Get the photon numbers distribution with alg. (\ref{alg:statistics_computation})\;
            Compute the loss $\mathcal C$ using eq. (\ref{eq:loss})\;
            \textbf{return} $\mathcal C$.
    \end{algorithm}

    \begin{algorithm}
        \caption{Training of the parameterized photonic circuit}\label{alg:training}
        \textbf{Parameters:} $N_\text{steps}\in\mathbb N$, initial parameters $\theta_0\in\mathbb R^K$, learning rate $\eta\in\mathbb R_+$

        \KwResult{Optimized hyperparameters $\theta\in\mathbb R^K$}\

        Initialize hyperparameters $\theta \gets \theta_0$\;
        \For{$i=1$ to $N_\text{steps}$}{
        Compute the loss $\mathcal C$ with alg. (\ref{alg:loss_computation})\;
        Compute the gradient $\nabla_\theta\mathcal C$ with shift rule and alg. (\ref{alg:loss_computation})\;
            Update the parameters $\theta \gets \theta - \eta\nabla_\theta\mathcal C$\;
        \textbf{end for}
        }
        \textbf{return} $\theta$.
    \end{algorithm}

\section{Results}
    We gather here the results of the simulations. We focus on the case of 2 QDOs. In particular we plot the profile of the binding energy as a function of the distance between the two nuclei. The binding energy is simply defined as the ground state energy of the interacting system to which one substract the ground state energy of the uninteracting system, namely that of a pair of free harmonic oscillators in this Drude model.

\section{Conclusion}

\begin{acknowledgments}

\end{acknowledgments}

\section*{Code Availability}

The reader will find an open source python code accompanying this paper following this \href{https://github.com/MatthieuSarkis/qdo}{github repository}.

\appendix

\nocite{*}
\bibliographystyle{IEEEtran}
\bibliography{bibliography}

\end{document}