\documentclass[reprint, amsmath, amssymb, aps]{revtex4-2}

\usepackage{graphicx}
\graphicspath{{./assets/figures/}}
\usepackage{dcolumn}
\usepackage{bm}
\usepackage{diagbox}
\usepackage[table]{xcolor}
\usepackage{hyperref}
\usepackage{comment}
\newcolumntype{L}{>{$}l<{$}} % math-mode version of "l" column type
\usepackage{float} % To fix location of tables with H
\usepackage[ruled,vlined]{algorithm2e}
\usepackage{braket}
\usepackage{amsthm}
\usepackage{cleveref}

\newtheorem{definition}{Definition}
\newtheorem{prop}{Proposition}
\DeclareMathOperator{\tr}{Tr}

\begin{document}

\preprint{}

\title{Simulating Quantum Drude Oscillators on a photonic quantum computer}
%\thanks{}

\author{Matthieu Sarkis}
\email{matthieu.sarkis@uni.lu}

\affiliation{
Department of Physics and Materials Science\\ University of Luxembourg, L-1511, Luxembourg City, Luxembourg.
}

\author{Adel Sohbi}
\email{sohbi@kias.re.kr}

\affiliation{
ORCA
}

%\collaboration{}%\noaffiliation

\date{\today}

\begin{abstract}
%\begin{description}
%\item[Usage]
%Secondary publications and information retrieval purposes.
%\item[Structure]
%You may use the \texttt{description} environment to structure your abstract;
%use the optional argument of the \verb+\item+ command to give the category of each item.
%\end{description}
\end{abstract}

%\keywords{Suggested keywords}%Use showkeys class option if keyword
                              %display desired
\maketitle

%\tableofcontents

\section{Introduction}


\section{Conclusion}

\begin{acknowledgments}

\end{acknowledgments}

\section*{Data Availability}

\section*{Code Availability}

The reader will find an open source python code accompanying this paper following this \href{https://github.com/MatthieuSarkis/qdo}{github repository}.

\appendix

\nocite{*}
\bibliographystyle{IEEEtran}
\bibliography{bibliography}

\end{document}