

%%%%%%%%%%%%%%%%%%%%%%%%%%%%%%%%%%%%%%%%%%%%%%%%%%%%
%%%%%%%%%%% UNUSED CODE %%%%%%%%%%%%%%%%%%%%%%%%%%%%
%%%%%%%%%%%%%%%%%%%%%%%%%%%%%%%%%%%%%%%%%%%%%%%%%%%%


Let us give here the expression for the multipolar potential up to quartic order:
        \begin{align}
            V_0(\bm{x} _i, \bm{x} _j) &= q_iq_j\frac{x_ix_j + y_iy_j - z_iz_j}{r_{ij}^3}\\
            V_1(\bm{x} _i, \bm{x} _j) &= \frac{q_iq_j}{2r_{ij}^4}\big(-3  x_i ^2  z_j -6  x_i   x_j   z_i +6  x_i   x_j   z_j +3  x_j ^2  z_i\nonumber\\
            & -3  y_i ^2  z_j -6  y_i   y_j   z_i +6  y_i   y_j   z_j +3  y_j ^2  z_i\nonumber\\
            & +6  z_i ^2  z_j -6  z_i   z_j ^2\big)\\
            V_2(\bm{x} _i, \bm{x} _j) &= \frac{q_iq_j}{2 r_{ij}^4}\big(-6  x_i ^3  x_j +9  x_i ^2  x_j ^2-6  x_i ^2  y_i   y_j +3  x_i ^2  y_j ^2\nonumber\\
            &+24  x_i ^2  z_i   z_j -12  x_i ^2  z_j ^2-6  x_i   x_j ^3-6  x_i   x_j   y_i ^2\nonumber\\
            &+12  x_i   x_j   y_i   y_j -6  x_i   x_j   y_j ^2+24  x_i   x_j   z_i ^2\nonumber\\
            &-48  x_i   x_j   z_i   z_j +24  x_i   x_j   z_j ^2+3  x_j ^2  y_i ^2-6  x_j ^2  y_i   y_j \nonumber\\
            &-12  x_j ^2  z_i ^2+24  x_j ^2  z_i   z_j -6  y_i ^3  y_j +9  y_i ^2  y_j ^2\nonumber\\
            &+24  y_i ^2  z_i   z_j -12  y_i ^2  z_j ^2-6  y_i   y_j ^3+24  y_i   y_j   z_i ^2\nonumber\\
            &-48  y_i   y_j   z_i   z_j +24  y_i   y_j   z_j ^2-12  y_j ^2  z_i ^2+24  y_j ^2  z_i   z_j \nonumber\\
            &-16  z_i ^3  z_j +24  z_i ^2  z_j ^2-16  z_i   z_j ^3\big)
        \end{align}



        \begin{equation}
            \begin{split}
                &V_0(x_i, x_j) = q_iq_j\frac{x_ix_j}{r_{ij}^3}\times\\
                &\times\begin{cases}
                    1-3\cos^2\theta, & \text{generic case} \\
                    -2, & \text{parallel case} \\
                    1, & \text{perpendicular case} \\
                    -\frac{1}{2}, & \text{oblique case} \\
                    -2-6\epsilon, & \text{regularized case}
                \end{cases}
            \end{split}
            \end{equation}
            The next terms in the multipolar expansion are:
            \begin{equation}
            \begin{split}
                &V_1(x_i, x_j) = q_iq_j\frac{x_ix_j(x_i-x_j)}{r_{ij}^4}\times\\
                &\times\begin{cases}
                    \frac{3\cos\theta(-3+5\cos^2\theta)}{2}, & \text{generic case} \\
                    3, & \text{parallel case} \\
                    0, & \text{perpendicular case} \\
                    -\frac{3}{4\sqrt 2}, & \text{oblique case} \\
                    3+18\epsilon, & \text{regularized case}
                \end{cases}
            \end{split}
            \end{equation}
            \begin{equation}
                \begin{split}
                    &V_2(x_i, x_j) = q_iq_j\frac{x_ix_j(2x_i^2-3x_ix_j+2x_j^2)}{r_{ij}^5}\times\\
                    &\times
                    \begin{cases}
                        -\frac{3-30\cos^2\theta+35\cos^4\theta}{4}, & \text{generic case} \\
                        -2, & \text{parallel case} \\
                        -\frac{3}{4}, & \text{perpendicular case} \\
                        \frac{13}{16}, & \text{oblique case} \\
                        -2-20\epsilon, & \text{regularized case}
                    \end{cases}
                \end{split}
            \end{equation}

            \subsection{12 different models}

            For the simulation, we will restrict the system to two QDOs. Depending on the space dimensionality (1d oblique, 1d parallel, 1d perpendicular, 3d) and the choice of potential (quadratic, quartic, Coulomb), we therefore have 12 models to try and study, gathered in table (\ref{tab:models}).
            \begin{table}[ht!]
            \caption{\label{tab:models} The twelve models}
            \begin{ruledtabular}
            \begin{tabular}{c|cccc}
                \diagbox[height=1.8\line]{\textsc{pot}}{\textsc{dim}}& 1d oblique & 1d parallel & 1d perpendicular & 3d \\
                \hline\\[-0.95em]
                \textsc{quad} & $H_{(1,0)}$ & $H_{(1,1)}$ & $H_{(1,2)}$ & $H_{(1,3)}$ \\
                \textsc{quart} & $H_{(2,0)}$ & $H_{(2,1)}$ & $H_{(2,2)}$ & $H_{(2,3)}$\\
                \textsc{Coulomb} & $H_{(3,0)}$ & $H_{(3,1)}$ & $H_{(3,2)}$ & $H$ \\
            \end{tabular}
            \end{ruledtabular}
            \end{table}

            The space regularized model Coulomb potential reads
            \begin{equation}
            \begin{split}
                &\frac{V_\text{Coul}^\epsilon(x_i, x_j)}{q_iq_j} = \frac{1}{r_{ij}} - \frac{1}{\sqrt{r_{ij}^2+2r_{ij}(1+\epsilon) x_i+x_i^2}} \\
                &- \frac{1}{\sqrt{r_{ij}^2-2r_{ij}(1+\epsilon) x_j+x_j^2}} \\
                &+\frac{1}{\sqrt{r_{ij}^2 - 2r_{ij}(1+\epsilon) (x_j-x_i) + (x_j-x_i)^2}}\,.
            \end{split}
            \end{equation}

            In terms of components, the full Coulomb potential reads:
            \begin{equation}
            \mathclap{
            \begin{split}
                &\frac{V_\text{Coul}(\bm{x} _i, \bm{x} _j)}{q_iq_j} = \frac{1}{r_{ij}} - \frac{1}{\sqrt{r_{ij}^2 + x_i^2+y_i^2+z_i^2+2rz_i}} \\
                &- \frac{1}{\sqrt{r_{ij}^2 + x_j^2+y_j^2+z_j^2-2r_{ij}z_j}} \\
                &+\frac{1}{\sqrt{r_{ij}^2 + (x_j-x_i)^2+(y_j-y_i)^2+(z_j-z_i)^2-2r_{ij}(z_j-z_i)}}
            \end{split}
            }
            \end{equation}

            For the full Coulomb potential, assuming that the drudons are constrained to move along an axis, we get the following expressions:
            \begin{equation}
            \begin{split}
                \frac{V_\text{Coul}^\perp(x_1, x_2)}{q_1q_2} = &\ \frac{1}{r_{12}} - \frac{1}{\sqrt{r_{12}^2+x_1^2}} - \frac{1}{\sqrt{r_{12}^2+x_2^2}} \\
                &+\frac{1}{\sqrt{r_{12}^2 + (x_2-x_1)^2}}
            \end{split}
            \end{equation}
            in the case where the drudons move perpendicular to the axis joining the two nuclei, and
            \begin{equation*}
            \mathclap{
                \frac{V_\text{Coul}^{||}(x_1, x_2)}{q_1q_2} = \frac{1}{r_{12}} - \frac{1}{|r_{12}+x_1|} - \frac{1}{|r_{12}-x_2|} + \frac{1}{|r_{12}+x_1-x_2|}
            }
            \end{equation*}
            in the case where they move parallel to the latter.


            \begin{equation}
                |\psi(\omega)\rangle = \sum_{n_1,\dots,n_{2K}=0}^\infty \alpha_{n_1\dots n_{2K}}|n_1\rangle\otimes\dots\otimes|n_{2K}\rangle\,.
            \end{equation}
            The modes labeled by $(n_1,\dots,n_K)$ correspond to $\text{QDO}_1$, while the modes $(n_{K+1},\dots,n_{2K})$ are attached to $\text{QDO}_2$.
            The amplitude of a specific tuple of the quadratures $(X_1,\dots, X_{K})$ is therefore given by:
            \begin{equation*}
            \mathclap{
                \langle X_1,\dots,X_{2K}|\psi\rangle = \sum_{n_1,\dots,n_{2K}=0}^\infty \alpha_{n_1\dots n_{2K}}\prod_{i=1}^{2K}\frac{e^{-\sum_{i=1}^{2K}\frac{X_i^2}{2}}H_{n_i}(X_i)}{\sqrt{\pi^{1/2}2^{n_i}n_i!}}\,,
            }
            \end{equation*}
            in terms of the Hermite polynomials. The joint law of the quadratures in the state $|\psi\rangle$ is therefore given by
            \begin{equation}
            \begin{split}
                \rho(X_1,\dots,X_{2K}) &= \sum_{\substack{n_1,\dots,n_{2K} \\ m_1,\dots,m_{2K}}} \alpha_{n_1\dots n_{2K}}\alpha^*_{m_1\dots m_{2K}}\\
                &\times\prod_{i=1}^{2K}\frac{e^{-X_i^2}H_{n_i}(X_i)H_{m_i}(X_i)}{\sqrt{\pi^{1/2}2^{n_i}n_i!}\sqrt{\pi^{1/2}2^{m_i}m_i!}}
            \end{split}
            \end{equation}
            After extracting as well the mean photon numbers $\langle n_{i,\alpha}\rangle$, one obtains $\langle H\rangle$.

            \begin{algorithm}
                \caption{Computation of the loss}\label{alg:loss_computation}
                    \textbf{Parameters:} $M\in\mathbb N$

                    \KwResult{Value of the loss $\mathcal C$}\

                    Initialize $\mathcal C \gets 0$\;
                    Get the position quadratures distribution with alg. (\ref{alg:statistics_computation})\;
                    Get the photon numbers distribution with alg. (\ref{alg:statistics_computation})\;
                    Compute the loss $\mathcal C$ using eq. (\ref{eq:loss})\;
                    \textbf{return} $\mathcal C$.
            \end{algorithm}

            \begin{equation}
                |\psi\rangle = \sum_{n_1,\dots,n_{2K}=0}^\infty \alpha_{n_1\dots n_{2K}}|n_1\rangle\otimes\dots\otimes|n_{2K}\rangle\,,
            \end{equation}
            leading to the following expression for the density matrix:
            \begin{equation*}
            \mathclap{
                \rho = \sum_{\substack{n_1,\dots,n_{2K} \\ m_1,\dots,m_{2K}}} \alpha^*_{m_1\dots m_{2K}}\alpha_{n_1\dots n_{2K}}|n_1\rangle\langle m_1|\otimes\dots\otimes|n_{2K}\rangle\langle m_{2K}|\,.
            }
            \end{equation*}
            The partial trace associated to $\text{QDO}_1$ is therefore given by:
            \begin{equation}
            \begin{split}
                \rho_1 = \sum_{\substack{n_1,\dots,n_{K} \\ m_1,\dots,m_{K} \\ l_1,\dots,l_{K}}}& \alpha^*_{m_1\dots m_{K}l_1,\dots,l_{K}}\alpha_{n_1\dots n_{K}l_1,\dots,l_{K}}\\
                &|n_1\rangle\langle m_1|\otimes\dots\otimes|n_{K}\rangle\langle m_{K}|\,.
            \end{split}
            \end{equation}